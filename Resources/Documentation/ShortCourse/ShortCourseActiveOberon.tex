\documentclass[a4wide,11pt]{article}
\usepackage{color,bbm}
\usepackage{longtable}
\usepackage{amssymb,amsmath,graphicx}
\usepackage{nonfloat}
\usepackage{colortbl}
\usepackage{fancybox}
\usepackage{hyperref}
\usepackage{listings}

% --------------------------------- page layout --------------------------------------
\pagestyle{headings}
% other font styles:
%\usepackage[math]{iwona} % iwona, kurier
%\usepackage{cmbright}

\oddsidemargin -0 cm
\evensidemargin 0 cm
%\topmargin -1.5cm
\textwidth 16   cm
%\textheight 25 cm

\parskip 5pt
\parindent 0cm

\definecolor{lightgrey}{rgb}{0.5,0.5,0.5}
\definecolor{darkgrey}{rgb}{0.4,0.4,0.4}

% --------------------------------- listings --------------------------------------


\newcommand{\changefont}[3]{\fontfamily{#1}\fontseries{#2}\fontshape{#3}\selectfont}

\newcommand{\progfont}{\changefont{pcr}{m}{n}}
\newcommand{\kwfont}{\changefont{pcr}{b}{n}}
\renewcommand{\lstlistingname}{Fig.}
\newcommand{\todo}[1]
            {\setlength{\fboxrule}{2pt}\fcolorbox{red}{yellow}{\begin{minipage}{\textwidth} \color{blue}$todo:$ #1 \end{minipage}}}

\newcommand{\pc}[1]{\makebox{\progfont #1}}
\newcommand{\kw}[1]{\makebox{\kwfont #1}}

\lstdefinelanguage{ebnf}[]{}
{morekeywords={},
sensitive=true,
comment=[l]{//},
comment=[s]{(*}{*)},
morestring=[b]',
morestring=[b]",
}
\lstdefinelanguage{Oberon}[]{Pascal}%
  {morekeywords={len,module},%
   sensitive=f,%
    morecomment=[s][\color{red}]{(*!}{*)}
  }[keywords]%
\lstset{language=Oberon,
basicstyle=\scriptsize\progfont,keywordstyle=\kwfont ,identifierstyle=\progfont,
commentstyle=\color{darkgrey}, stringstyle=, showstringspaces=false, %keepspaces=true,
numbers=none, numberstyle=\tiny, stepnumber=1, numbersep=5pt, captionpos=b,
columns=flexible % flexible, fixed, fullflexible
framerule=1mm,frame=shadowbox, rulesepcolor=\color{blue}, % frame = shadowbox
xleftmargin=2mm,xrightmargin=2mm,
}
\renewcommand{\lstlistingname}{Fig.}

\begin{document}
\title{Programming in Active Oberon \\[1em] \normalsize Short Introductory Course}
\author{Sven Stauber, Ulrike Glavitsch, Felix Friedrich}
\maketitle

\tableofcontents

% --------------------------------------- example -------------------------------
\section{Section}
\subsection{Subsection}
\subsubsection{SubSubSection}
\paragraph{Paragraph}

Example Listing EBNF
\begin{lstlisting}[language=ebnf]
Number       =  Integer | Real.
Integer      =  Digit {Digit} | Digit {HexDigit} 'H'.
Real         =  Digit {Digit} '.' {Digit} [ScaleFactor].
ScaleFactor  =  ('E' | 'D') ['+' | '-'] digit {digit}.
HexDigit     =  Digit | 'A' | 'B' | 'C' | 'D' | 'E' | 'F'.
Digit = '0' | '1' | '2' | '3' | '4' | '5' | '6' | '7' | '8' | '9' .
\end{lstlisting}

Example Listing EBNF no frame
\begin{lstlisting}[language=ebnf,frame=none]
Number       =  Integer | Real.
Integer      =  Digit {Digit} | Digit {HexDigit} 'H'.
\end{lstlisting}


Example Listing Oberon
\begin{lstlisting}[language=Oberon]
    MODULE Test;
    PROCEDURE Test;
    VAR a,b,c: LONGINT;
    BEGIN
        a := b*b (*! b is not initialized *) 
    END Test;
    END Test.
\end{lstlisting}

Example Listing Oberon no frame with line numbers
\begin{lstlisting}[language=Oberon,frame=none,numbers=left]
    MODULE Test; (* module in upper case *)
    PROCEDURE Test;
    VAR a,b,c: LONGINT;
    BEGIN
        a := b*b
    END Test;
    END Test.
\end{lstlisting}

Example Listing Oberon NoFrame
\begin{lstlisting}[language=Oberon,frame=none]
    module Test; (* module in lower case *)
    procedure Test;
    var a,b,c: longint;
    begin
        a := b*b
    end Test;
    end Test.
\end{lstlisting}

Oberon keyword \kw{module} in text.

Oberon programming content \verb+a := b*b+ in text.

Math content $x \leftarrow 10$ in text.

Math content alone:
$$x \leftarrow 10$$

% --------------------------------------- end of example --------------------------

\section{A First Oberon Program}

\section{Basic Types}

\section{In- and Output}

\section{The module System}

\section{Active Objects}

\section{Active Oberon EBNF}


\end{document}

